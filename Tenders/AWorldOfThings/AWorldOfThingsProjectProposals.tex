\documentclass{article}

\usepackage{lipsum}
\usepackage[margin=2cm, left=2cm, includefoot]{geometry}
\usepackage{graphicx}
\usepackage{float}
\usepackage{hyperref}

% Header and footer
\usepackage{fancyhdr}
\pagestyle{fancy}

\rhead{}
\lhead{}
\fancyfoot{}
\fancyfoot[R]{\thepage}
\renewcommand{\headrulewidth}{0pt}
\renewcommand{\footrulewidth}{0pt}
%

\begin{document}

\begin{titlepage}
	\begin{center}
		\line(1,0){400}\\
		[6mm]
		\huge{\bfseries PROJECT PROPOSALS\\A WORLD OF THINGS}\\
		\line(1,0){400}\\
	\end{center}
\end{titlepage}

\section{A World of Plants}
	\subsection{Intro}
		This project involves the networking and monitoring of living plants so as to be able to analyse aspects of their environment such as water intake, lighting, mineral composition and temperature. These results will be used to create a collaborative platform on which users may learn from them so as to grow the best possible plants. The platform's intended use is to encourage younger generations to become interested in agriculture so as to stimulate South Africa's agricultural industry. We plan to implement gamification on our platform as a way to encourage competition amongst users. Given enough time we would like to use AI to learn how to analyse the growth of a plant so that we may automate the process. Through doing this we would like to create an even greater challenge, which is for users to attempt to grow a plant better than our control plant grown with AI.
	\subsection{What We Plan to Deliver}
		\begin{itemize}
			\item
		\end{itemize}
	\subsection{Additional Goals Depending on Time}
	\subsection{Hardware Requirements}
	\subsection{Amazon Web Services Requirements}
	\subsection{Proposal of Technologies to Use}
\end{document}

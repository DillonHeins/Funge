\documentclass{article}

\usepackage{lipsum}
\usepackage[margin=2cm, left=2cm, includefoot]{geometry}
\usepackage{graphicx}
\usepackage{float}
\usepackage{hyperref}

% Header and footer
\usepackage{fancyhdr}
\pagestyle{fancy}

\rhead{}
\lhead{}
\fancyfoot{}
\fancyfoot[R]{\thepage}
\renewcommand{\headrulewidth}{0pt}
\renewcommand{\footrulewidth}{0pt}
%

\begin{document}

\begin{titlepage}
	\begin{center}
		\line(1,0){500}\\
		[6mm]
		\huge{\bfseries Vision, Scope, Architectural Requirements and\\Software Architecture Design}\\
		\line(1,0){500}\\
		[5mm]
		\large\textbf{Project:}\\A World of Things\\
		[3mm]
		\large\textbf{Client:}\\Julian Hambleton-Jones\\
		[3mm]
		\large \textbf{Team:}\\Funge\\
		\line(1,0){500}\\
		[5mm]
		\large \textbf{Team Members:}\\
		[3mm]
		\large 14214742 - Matthew Botha\\
		\large 14446619 - Gian Paolo Buffo\\
		\large 14027021 - Matthias Harvey\\
        \large 14035538 - Dillon Heins\\[3mm]
	\end{center}
\end{titlepage}

\cleardoublepage
\thispagestyle{empty}
\tableofcontents
\cleardoublepage
\setcounter{page}{1}

\section{Vision}
The aim of this project is to build an innovative Internet of Things solution through the use of the Amazon Web Services Internet of Things hardware platform as well as the Amazon Web Services Cloud. No problem was formally defined and hence it was up to us to determine what solution we would be creating. We opted to create a solution which focuses on the education of students with regard to agriculture and plant sciences.\\\\
We plan to motivate students to become interested in agriculture by creating an "Internet of Plants". This involves the networking and monitoring of living plants in order to analyse aspects of their environment - such as water intake, lighting, humidity, moisture, mineral composition and temperature. These results will be used to create a collaborative platform on which users may share their own results and learn from the results of others, in order to grow the best possible plants. The main purpose of the platform is to encourage younger generations to become interested in agriculture and, in doing so, help stimulate South Africa's agricultural industry.\\\\
We plan to implement gamification on our platform as a way to encourage competition amongst users. Given enough time, we would like to automate more parts of the system - such as automated watering. Additionally, AI learning could be employed to optimise the conditions under which plants grow. By combining automation and AI learning, we could create an interesting challenge for the
users: Grow a plant better than our control plant grown with the help of AI.
\cleardoublepage

\section{Scope}

\cleardoublepage

\section{Architectural Requirements}
\subsection{Access and Integration Requirements}
\subsection{Quality Requirements}
\subsection{Architectural Responsibilities}
\subsection{Architecture Constraints}

\cleardoublepage

\section{Architecture Design}
\subsection{Architectural Tactics}
\subsection{Architectural Components Addressing Architectural Responsibilities}
\subsection{Infrastructure}
\subsection{Concepts and Constraints for Application Components}
\end{document}

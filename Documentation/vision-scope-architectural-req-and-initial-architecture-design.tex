\documentclass{article}

\usepackage{lipsum}
\usepackage[margin=2cm, left=2cm, includefoot]{geometry}
\usepackage{graphicx}
\usepackage{float}
\usepackage{hyperref}

% Header and footer
\usepackage{fancyhdr}
\pagestyle{fancy}

\rhead{}
\lhead{}
\fancyfoot{}
\fancyfoot[R]{\thepage}
\renewcommand{\headrulewidth}{0pt}
\renewcommand{\footrulewidth}{0pt}
%

\begin{document}

\begin{titlepage}
	\begin{center}
		\line(1,0){500}\\
		[6mm]
		\huge{\bfseries Vision, Scope, Architectural Requirements and\\Software Architecture Design}\\
		\line(1,0){500}\\
		[5mm]
		\large\textbf{Project:}\\A World of Things\\
		[3mm]
		\large\textbf{Client:}\\Julian Hambleton-Jones\\
		[3mm]
		\large \textbf{Team:}\\Funge\\
		\line(1,0){500}\\
		[5mm]
		\large \textbf{Team Members:}\\
		[3mm]
		\large 14214742 - Matthew Botha\\
		\large 14446619 - Gian Paolo Buffo\\
		\large 14027021 - Matthias Harvey\\
        \large 14035538 - Dillon Heins\\[3mm]
	\end{center}
\end{titlepage}

\cleardoublepage
\thispagestyle{empty}
\tableofcontents
\cleardoublepage
\setcounter{page}{1}

\section{Vision}
The aim of this project is to build an innovative Internet of Things solution through the use of the Amazon Web Services Internet of Things hardware platform as well as the Amazon Web Services Cloud. No problem was formally defined and hence it was up to us to determine what solution we would be creating. We opted to create a solution which focuses on the education of students with regard to agriculture and plant sciences.\\\\
We plan to motivate students to become interested in agriculture by creating an "Internet of Plants". This involves the networking and monitoring of living plants in order to analyse aspects of their environment - such as water intake, lighting, humidity, moisture, mineral composition and temperature. These results will be used to create a collaborative platform on which users may share their own results and learn from the results of others, in order to grow the best possible plants. The main purpose of the platform is to encourage younger generations to become interested in agriculture and, in doing so, help stimulate South Africa's agricultural industry.\\\\
We plan to implement gamification on our platform as a way to encourage competition amongst users. Given enough time, we would like to automate more parts of the system - such as automated watering. Additionally, AI learning could be employed to optimise the conditions under which plants grow. By combining automation and AI learning, we could create an interesting challenge for the
users: Grow a plant better than our control plant grown with the help of AI.
\cleardoublepage

\section{Scope}

\cleardoublepage

\section{Architectural Requirements}
\subsection{Access and Integration Requirements}
\subsection{Quality Requirements}

\subsubsection{Maintainability}
The system should be designed and created in such a way that it easy to maintain in the future.
\begin{itemize}
	\item New developers should be able to easily understand the system
	\item It should be easy to modify parts of the system
	\item Adding new functionality should be easy and straightforward
\end{itemize}
\subsubsection{Flexibility}
It is important that the system has support for many hardware components such as sensors. These technologies are advancing rapidly, therefore it should be easy to add new components to the system, without making any large changes.
\subsubsection{Extensibility}
a program that it easy to modify parts of living plants in order to motivate students to model concurrent systems formally defined and learn from the
system dding new components such as sensors hese results of their own 
\subsubsection{Performance}
\subsubsection{Scalability}
\subsubsection{Security}
\subsubsection{Auditabilty}
\subsubsection{Usability}
\subsubsection{Testability}

\subsection{Architectural Responsibilities}
\subsection{Architecture Constraints}

\cleardoublepage

\section{Architecture Design}
\subsection{Architectural Tactics}
\subsection{Architectural Components Addressing Architectural Responsibilities}
\subsection{Infrastructure}
	We will be using the Amazon WebServices SDK and APIs as we need to use the AWS IoT (Internet of Things) platform for the project. As such, we will be using other AWS products together with IoT due to their seamless intergration.
		\begin{figure}[H]
			\centering
			\includegraphics[width=\textwidth]{AWSMap.png}
			\caption{Communication between AWS services}
		\end{figure}
	\subsubsection{Internet of Things}
		AWS IoT is a managed cloud platform that lets connected devices easily and securely interact with cloud applications and other devices.
		
		We will be using the IoT platform to connect to Lambda (explained below) using the MQTT protocol. As part of the project, we are required to use the AWS IoT platform.
	\subsubsection{Lambda}
		AWS Lambda lets us run code without provisioning or managing servers. You pay only for the compute time you consume - there is no charge when your code is not running. With Lambda, you can run code for virtually any type of application or backend service - all with zero administration. We can just upload our code and Lambda takes care of everything required to run and scale our code with high availability. We can set up your code to automatically trigger from other AWS services or call it directly from any web or mobile app.
		
		Lambda will form the largest part of our backend. All messages will be sent from IoT to Lambda, using MQTT, and that event will trigger a response and the web interface can access Lambda through the API Gateway, negating the need for a server to be constantly listening. Lambda can then perform operations on DynamoDB, but can also respond to events triggered when data in DynamoDB is changed or added. 
	\subsubsection{DynamoDB}
		Amazon DynamoDB is a fast and flexible NoSQL database service which has built in Java support. Because it is a NoSQL database, we can push and pull Java objects using the API. It intergrates with AWS Lambda, allowing Lambda to respond to events in the database. It can also be easily accessed through the API Gateway, allowing for fast read-write operations from the client interface.
	\subsubsection{API Gateway}
		Amazon API Gateway is a fully managed service that makes it easy for developers to create, publish, maintain, monitor, and secure APIs at any scale. We can easily and quickly create an API that acts as a “front door” for the user interface to access data from DynamoDB or trigger events in Lambda by exposing RESTful services to the API. Amazon API Gateway handles all the tasks involved in accepting and processing up to hundreds of thousands of concurrent API calls, including traffic management, authorization and access control, monitoring, and API version management.
	\subsubsection{JavaScript SDK}
		The AWS SDK for JavaScript enables us to directly access AWS services from JavaScript code running in the browser. We can authenticate users, store data in DynamoDB and trigger Lambda events. It is simple to deploy and uses the same API as the other components, meaning function migration becomes trivial.
\subsection{Concepts and Constraints for Application Components}
\end{document}
